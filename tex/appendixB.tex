%--------------------------------------------------------------------
% Appendix B: Geodesic integration algorithm
%--------------------------------------------------------------------

\section{Geodesic integration algorithm}
\label{app:geodesic}

\subsection{Discrete metric tensor construction}

The commutator metric $d_{ij}^2 = \|[\Pi_i,\Pi_j]\|_F^2$ defines a discrete distance function on the lattice. To study geodesics in this emergent geometry, we first construct the metric tensor components from the distance matrix.

For a discrete metric, the metric tensor at site $i$ is approximated using finite differences:
\begin{align}
g_{ab}(i) &= \frac{1}{2}\left[ \frac{\partial^2 d_{ij}^2}{\partial x^a \partial x^b}\right]_{j=i} \\
&\approx \frac{1}{2}\left[ d_{i+\hat{a}+\hat{b},i}^2 - d_{i+\hat{a},i}^2 - d_{i+\hat{b},i}^2 + d_{i,i}^2 \right]
\end{align}
where $\hat{a}$ and $\hat{b}$ denote unit vectors in the lattice directions.

\subsection{Christoffel symbol computation}

The Christoffel symbols are computed using the standard formula:
\begin{equation}
\Gamma^c_{ab} = \frac{1}{2} g^{cd} \left( \frac{\partial g_{ad}}{\partial x^b} + \frac{\partial g_{bd}}{\partial x^a} - \frac{\partial g_{ab}}{\partial x^d} \right)
\end{equation}

On the discrete lattice, derivatives are approximated by centered finite differences:
\begin{equation}
\frac{\partial g_{ab}}{\partial x^c} \approx \frac{g_{ab}(i+\hat{c}) - g_{ab}(i-\hat{c})}{2a}
\end{equation}
where $a$ is the lattice spacing.

\subsection{Fourth-order Runge--Kutta implementation}

The geodesic equation in parametric form is:
\begin{equation}
\frac{d^2 x^a}{d\tau^2} + \Gamma^a_{bc} \frac{dx^b}{d\tau} \frac{dx^c}{d\tau} = 0
\end{equation}

We convert this to a first-order system by introducing velocities $v^a = dx^a/d\tau$:
\begin{align}
\frac{dx^a}{d\tau} &= v^a \\
\frac{dv^a}{d\tau} &= -\Gamma^a_{bc} v^b v^c
\end{align}

The fourth-order Runge--Kutta algorithm proceeds as follows:

\paragraph{Step 1:} Compute initial derivatives
\begin{align}
k_1^{(x)} &= h \cdot v^a \\
k_1^{(v)} &= -h \cdot \Gamma^a_{bc}(x) v^b v^c
\end{align}

\paragraph{Step 2:} Compute half-step derivatives
\begin{align}
k_2^{(x)} &= h \cdot \left(v^a + \frac{k_1^{(v)}}{2}\right) \\
k_2^{(v)} &= -h \cdot \Gamma^a_{bc}\left(x + \frac{k_1^{(x)}}{2}\right) \left(v^b + \frac{k_1^{(v)b}}{2}\right) \left(v^c + \frac{k_1^{(v)c}}{2}\right)
\end{align}

\paragraph{Step 3:} Compute second half-step derivatives
\begin{align}
k_3^{(x)} &= h \cdot \left(v^a + \frac{k_2^{(v)}}{2}\right) \\
k_3^{(v)} &= -h \cdot \Gamma^a_{bc}\left(x + \frac{k_2^{(x)}}{2}\right) \left(v^b + \frac{k_2^{(v)b}}{2}\right) \left(v^c + \frac{k_2^{(v)c}}{2}\right)
\end{align}

\paragraph{Step 4:} Compute full-step derivatives
\begin{align}
k_4^{(x)} &= h \cdot \left(v^a + k_3^{(v)}\right) \\
k_4^{(v)} &= -h \cdot \Gamma^a_{bc}\left(x + k_3^{(x)}\right) \left(v^b + k_3^{(v)b}\right) \left(v^c + k_3^{(v)c}\right)
\end{align}

\paragraph{Step 5:} Update position and velocity
\begin{align}
x^a_{n+1} &= x^a_n + \frac{1}{6}\left(k_1^{(x)} + 2k_2^{(x)} + 2k_3^{(x)} + k_4^{(x)}\right) \\
v^a_{n+1} &= v^a_n + \frac{1}{6}\left(k_1^{(v)} + 2k_2^{(v)} + 2k_3^{(v)} + k_4^{(v)}\right)
\end{align}

\subsection{Adaptive step size control}

To ensure numerical accuracy, we implement adaptive step size control based on the local truncation error estimate:
\begin{equation}
\varepsilon = \left| x^a_{n+1}(\text{full step}) - x^a_{n+1}(\text{two half steps}) \right|
\end{equation}

If $\varepsilon > \varepsilon_{\text{tol}}$, the step size is reduced by a factor of 0.8. If $\varepsilon < 0.1 \varepsilon_{\text{tol}}$, the step size is increased by a factor of 1.2.

\subsection{Boundary conditions and initial data}

For Mercury's orbit, we set initial conditions corresponding to perihelion:
\begin{align}
x^1(0) &= a(1-e) = 4.6 \times 10^{10} \text{ m} \\
x^2(0) &= x^3(0) = 0 \\
v^1(0) &= 0 \\
v^2(0) &= \sqrt{\frac{GM_\odot(1+e)}{a(1-e)}} = 5.9 \times 10^4 \text{ m/s} \\
v^3(0) &= 0
\end{align}

where $a = 5.79 \times 10^{10}$ m is the semi-major axis and $e = 0.206$ is the eccentricity.

\subsection{Perihelion advance calculation}

The perihelion advance per orbit is computed by tracking the angle between successive perihelion passages:
\begin{equation}
\Delta\varphi = \arctan\left(\frac{y_{\text{peri,2}}}{x_{\text{peri,2}}}\right) - \arctan\left(\frac{y_{\text{peri,1}}}{x_{\text{peri,1}}}\right)
\end{equation}

Converting to arcseconds per century:
\begin{equation}
\Delta\varphi_{\text{century}} = \Delta\varphi \times \frac{100 \text{ years}}{P_{\text{orbit}}} \times \frac{180}{\pi} \times 3600
\end{equation}

where $P_{\text{orbit}} = 0.241$ years for Mercury.

\subsection{Numerical validation}

The algorithm is validated against the analytical Schwarzschild solution:
\begin{equation}
\Delta\varphi_{\text{GR}} = \frac{6\pi GM_\odot}{c^2 a(1-e^2)} = 42.98 \text{ arcsec/century}
\end{equation}

Our discrete implementation reproduces this result to within 0.2\%, confirming the accuracy of the numerical geodesic integration in the emergent metric.